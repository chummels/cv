% The formatting of this CV is based on @davidwhogg's layout.

\documentclass[12pt,letterpaper]{article}

\usepackage{color}
\usepackage{fancyhdr}
\usepackage{hyperref}
\usepackage{ifthen}

% \usepackage[yyyymmdd]{datetime}
% \renewcommand{\dateseparator}{-}

% Link formatting.
\definecolor{numcolor}{rgb}{0.5,0.5,0.5}
\definecolor{linkcolor}{rgb}{0,0,0.4}
\hypersetup{%
    colorlinks=true,        % false: boxed links; true: colored links
    linkcolor=linkcolor,    % color of internal links
    citecolor=linkcolor,    % color of links to bibliography
    filecolor=linkcolor,    % color of file links
    urlcolor=linkcolor      % color of external links
}

% Text formatting.
\newcommand{\foreign}[1]{\textit{#1}}
\newcommand{\etal}{\foreign{et~al.}}
\newcommand{\project}[1]{\textsl{#1}}
\definecolor{grey}{rgb}{0.5,0.5,0.5}
\newcommand{\deemph}[1]{\textcolor{grey}{\footnotesize{#1}}}

% literature links--use doi if you can
  \newcommand{\doi}[2]{\emph{\href{http://dx.doi.org/#1}{{#2}}}}
  \newcommand{\ads}[2]{\href{http://adsabs.harvard.edu/abs/#1}{{#2}}}
  \newcommand{\isbn}[1]{{\footnotesize(\textsc{isbn:}{#1})}}
  \newcommand{\arxiv}[1]{{\href{http://arxiv.org/abs/#1}{arXiv:{#1}}}}

% Section headings.
\newcommand{\cvheading}[1]{\addvspace{1ex}\pagebreak[2]\par\textbf{#1}\nopagebreak\vspace{-0.4em}}

% Set up the custom unordered list.
\newcounter{refpubnum}
\newcommand{\cvlist}{%
    \rightmargin=0in
    \leftmargin=0.15in
    \topsep=0ex
    \partopsep=0pt
    \itemsep=0.2ex
    \parsep=0pt
    \itemindent=-1.0\leftmargin
    \listparindent=0.0\leftmargin
    \settowidth{\labelsep}{~}
    \usecounter{refpubnum}
}

% Margins and spaces.
\raggedright
\setlength{\oddsidemargin}{0in}
\setlength{\topmargin}{0in}
\setlength{\headsep}{0.20in}
\setlength{\headheight}{0.25in}
\setlength{\textheight}{9.1in}
\addtolength{\topmargin}{-\headsep}
\addtolength{\topmargin}{-\headheight}
\setlength{\textwidth}{6.50in}
\setlength{\parindent}{0in}
\setlength{\parskip}{1ex}

% Headings and footings.
\renewcommand{\headrulewidth}{0pt}
\pagestyle{fancy}
\lhead{\deemph{Daniel Foreman-Mackey}}
\chead{\deemph{Curriculum Vitae}}
\rhead{\deemph{\thepage}}
\cfoot{\deemph{Last updated: \today}}

% Journal names.
\newcommand{\aj}{AJ}
\newcommand{\apj}{ApJ}
\newcommand{\pasp}{PASP}
\newcommand{\mnras}{MNRAS}


\begin{document}\thispagestyle{empty}\sloppy\sloppypar\raggedbottom

\textbf{\Large Daniel Foreman-Mackey} \hfill
\textsf{\small foreman.mackey@gmail.com, http://dan.iel.fm} \\[0.5ex]
Associate Research Scientist, Center for Computational Astronomy, Flatiron Institute\\[0.5ex]

\cvheading{Education}
\begin{list}{}{\cvlist}
\item
PhD 2015, Department of Physics, New York University. Advisor: Hogg
\item
MSc 2010, Department of Physics, Queen's University, Canada. Advisor: Widrow
\item
BSc 2008, Department of Physics, McGill University, Canada.
\end{list}

\cvheading{Positions}
\begin{list}{}{\cvlist}
\item
Associate Research Scientist, Flatiron Institute, 2017--present.
\item
Sagan Postdoctoral Fellow, University of Washington, 2015--2017.
\end{list}

\ifdefined\withpubs
    \cvheading{Publications}
    \textbf{Total:} 11 $\smash{|}$ \textbf{Refereed:} 8 $\smash{|}$ \textbf{First Author:} 7 $\smash{|}$ \textbf{Citations:} 237 $\smash{|}$ \textbf{h-index:} 9

    \cvheading{Refereed publications}
    \begin{list}{}{\cvlist}
    \item[{\color{numcolor}\scriptsize27}] Price-Whelan, Adrian M., \& \textbf{Foreman-Mackey, Daniel}, 2017, \doi{10.21105/joss.00357}{schwimmbad: A uniform interface to parallel processing pools in Python}, The Journal of Open Source Software, \textbf{2}, 17

\item[{\color{numcolor}\scriptsize26}] Luger, Rodrigo, Sestovic, Marko, Kruse, Ethan, Grimm, Simon L., \etal\ (incl.\ \textbf{DFM}), 2017, \doi{10.1038/s41550-017-0129}{A seven-planet resonant chain in TRAPPIST-1}, Nature Astronomy, \textbf{1}, 129 (\arxiv{1703.04166}) [\href{http://adsabs.harvard.edu/abs/2017NatAs...1E.129L}{11 citations}]

\item[{\color{numcolor}\scriptsize25}] Price-Whelan, Adrian M., Hogg, David W., \textbf{Foreman-Mackey, Daniel}, \& Rix, Hans-Walter, 2017, \doi{10.3847/1538-4357/aa5e50}{The Joker: A Custom Monte Carlo Sampler for Binary-star and Exoplanet Radial Velocity Data}, \apj, \textbf{837}, 20 (\arxiv{1610.07602}) [\href{http://adsabs.harvard.edu/abs/2017ApJ...837...20P}{2 citations}]

\item[{\color{numcolor}\scriptsize24}] Henderson, Calen B., Poleski, Rados{\l}aw, Penny, Matthew, Street, Rachel A., \etal\ (incl.\ \textbf{DFM}), 2016, \doi{10.1088/1538-3873/128/970/124401}{Campaign 9 of the K2 Mission: Observational Parameters, Scientific Drivers, and Community Involvement for a Simultaneous Space- and Ground-based Microlensing Survey}, \pasp, \textbf{128}, 124401 (\arxiv{1512.09142}) [\href{http://adsabs.harvard.edu/abs/2016PASP..128l4401H}{27 citations}]

\item[{\color{numcolor}\scriptsize23}] Hogg, David W., Casey, Andrew R., Ness, Melissa, Rix, Hans-Walter, \etal\ (incl.\ \textbf{DFM}), 2016, \doi{10.3847/1538-4357/833/2/262}{Chemical Tagging Can Work: Identification of Stellar Phase-space Structures Purely by Chemical-abundance Similarity}, \apj, \textbf{833}, 262 (\arxiv{1601.05413}) [\href{http://adsabs.harvard.edu/abs/2016ApJ...833..262H}{14 citations}]

\item[{\color{numcolor}\scriptsize22}] \textbf{Foreman-Mackey, Daniel}, Morton, Timothy D., Hogg, David W., Agol, Eric, \& Sch{\"o}lkopf, Bernhard, 2016, \doi{10.3847/0004-6256/152/6/206}{The Population of Long-period Transiting Exoplanets}, \aj, \textbf{152}, 206 (\arxiv{1607.08237}) [\href{http://adsabs.harvard.edu/abs/2016AJ....152..206F}{10 citations}]

\item[{\color{numcolor}\scriptsize21}] Angus, Ruth, Aigrain, Susanne, \& \textbf{Foreman-Mackey, Daniel}, 2016, \doi{10.1017/S1743921316002738}{Stellar rotation period inference with Gaussian processes}, IAU Focus Meeting, \textbf{29A}, 191

\item[{\color{numcolor}\scriptsize20}] Luger, Rodrigo, Agol, Eric, Kruse, Ethan, Barnes, Rory, \etal\ (incl.\ \textbf{DFM}), 2016, \doi{10.3847/0004-6256/152/4/100}{EVEREST: Pixel Level Decorrelation of K2 Light Curves}, \aj, \textbf{152}, 100 (\arxiv{1607.00524}) [\href{http://adsabs.harvard.edu/abs/2016AJ....152..100L}{28 citations}]

\item[{\color{numcolor}\scriptsize19}] Wang, Dun, Hogg, David W., \textbf{Foreman-Mackey, Daniel}, \& Sch{\"o}lkopf, Bernhard, 2016, \doi{10.1088/1538-3873/128/967/094503}{A Causal, Data-driven Approach to Modeling the Kepler Data}, \pasp, \textbf{128}, 94503 (\arxiv{1508.01853}) [\href{http://adsabs.harvard.edu/abs/2016PASP..128i4503W}{4 citations}]

\item[{\color{numcolor}\scriptsize18}] Fischer, Debra A., Anglada-Escude, Guillem, Arriagada, Pamela, Baluev, Roman V., \etal\ (incl.\ \textbf{DFM}), 2016, \doi{10.1088/1538-3873/128/964/066001}{State of the Field: Extreme Precision Radial Velocities}, \pasp, \textbf{128}, 66001 (\arxiv{1602.07939}) [\href{http://adsabs.harvard.edu/abs/2016PASP..128f6001F}{39 citations}]

\item[{\color{numcolor}\scriptsize17}] \textbf{Foreman-Mackey, Daniel}, 2016, \doi{10.21105/joss.00024}{corner.py: Scatterplot matrices in Python}, The Journal of Open Source Software, \textbf{1}, 2 [\href{https://scholar.google.com/scholar?cites=1835087844145558435,17836006976722650130,13220763673061560856}{117 citations}]

\item[{\color{numcolor}\scriptsize16}] Sch{\"o}lkopf, Bernhard, Hogg, David W., Wang, Dun, \textbf{Foreman-Mackey, Daniel}, \etal, 2016, \doi{10.1073/pnas.1511656113}{Modeling confounding by half-sibling regression}, PNAS, \textbf{113}, 27 [\href{https://scholar.google.com/scholar?cites=2429561747341807338}{2 citations}]

\item[{\color{numcolor}\scriptsize15}] Angus, Ruth, \textbf{Foreman-Mackey, Daniel}, \& Johnson, John A., 2016, \doi{10.3847/0004-637X/818/2/109}{Systematics-insensitive Periodic Signal Search with K2}, \apj, \textbf{818}, 109 (\arxiv{1505.07105}) [\href{http://adsabs.harvard.edu/abs/2016ApJ...818..109A}{12 citations}]

\item[{\color{numcolor}\scriptsize14}] Ambikasaran, Sivaram, \textbf{Foreman-Mackey, Daniel}, Greengard, Leslie, Hogg, David W., \& O'Neil, Michael, 2016, \doi{10.1109/TPAMI.2015.2448083}{Fast Direct Methods for Gaussian Processes}, IEEE Transactions on Pattern Analysis and Machine Intelligence, \textbf{38}, 252 (\arxiv{1403.6015}) [\href{https://scholar.google.com/scholar?cites=4840899390891567426,9641158393712381489}{70 citations}]

\item[{\color{numcolor}\scriptsize13}] Montet, Benjamin T., Morton, Timothy D., \textbf{Foreman-Mackey, Daniel}, Johnson, John Asher, \etal, 2015, \doi{10.1088/0004-637X/809/1/25}{Stellar and Planetary Properties of K2 Campaign 1 Candidates and Validation of 17 Planets, Including a Planet Receiving Earth-like Insolation}, \apj, \textbf{809}, 25 (\arxiv{1503.07866}) [\href{http://adsabs.harvard.edu/abs/2015ApJ...809...25M}{35 citations}]

\item[{\color{numcolor}\scriptsize12}] Barclay, Thomas, Quintana, Elisa V., Adams, Fred C., Ciardi, David R., \etal\ (incl.\ \textbf{DFM}), 2015, \doi{10.1088/0004-637X/809/1/7}{The Five Planets in the Kepler-296 Binary System All Orbit the Primary: A Statistical and Analytical Analysis}, \apj, \textbf{809}, 7 (\arxiv{1505.01845}) [\href{http://adsabs.harvard.edu/abs/2015ApJ...809....7B}{16 citations}]

\item[{\color{numcolor}\scriptsize11}] Angus, Ruth, Aigrain, Suzanne, \textbf{Foreman-Mackey, Daniel}, \& McQuillan, Amy, 2015, \doi{10.1093/mnras/stv423}{Calibrating gyrochronology using Kepler asteroseismic targets}, \mnras, \textbf{450}, 1787 (\arxiv{1502.06965}) [\href{http://adsabs.harvard.edu/abs/2015MNRAS.450.1787A}{30 citations}]

\item[{\color{numcolor}\scriptsize10}] Weisz, Daniel R., Johnson, L. Clifton, \textbf{Foreman-Mackey, Daniel}, Dolphin, Andrew E., \etal, 2015, \doi{10.1088/0004-637X/806/2/198}{The High-mass Stellar Initial Mass Function in M31 Clusters}, \apj, \textbf{806}, 198 (\arxiv{1502.06621}) [\href{http://adsabs.harvard.edu/abs/2015ApJ...806..198W}{18 citations}]

\item[{\color{numcolor}\scriptsize9}] \textbf{Foreman-Mackey, Daniel}, Montet, Benjamin T., Hogg, David W., Morton, Timothy D., \etal, 2015, \doi{10.1088/0004-637X/806/2/215}{A Systematic Search for Transiting Planets in the K2 Data}, \apj, \textbf{806}, 215 (\arxiv{1502.04715}) [\href{http://adsabs.harvard.edu/abs/2015ApJ...806..215F}{57 citations}]

\item[{\color{numcolor}\scriptsize8}] Sch{\"o}lkopf, Bernhard, Hogg, David W., Wang, Dun, \textbf{Foreman-Mackey, Daniel}, \etal, 2015, Removing systematic errors for exoplanet search via latent causes, ICML, \textbf{37}, 2218 (\arxiv{1505.03036}) [\href{https://scholar.google.com/scholar?cites=11768165421845046384}{2 citations}]

\item[{\color{numcolor}\scriptsize7}] Barclay, Thomas, Endl, Michael, Huber, Daniel, \textbf{Foreman-Mackey, Daniel}, \etal, 2015, \doi{10.1088/0004-637X/800/1/46}{Radial Velocity Observations and Light Curve Noise Modeling Confirm that Kepler-91b is a Giant Planet Orbiting a Giant Star}, \apj, \textbf{800}, 46 (\arxiv{1408.3149}) [\href{http://adsabs.harvard.edu/abs/2015ApJ...800...46B}{23 citations}]

\item[{\color{numcolor}\scriptsize6}] \textbf{Foreman-Mackey, Daniel}, Hogg, David W., \& Morton, Timothy D., 2014, \doi{10.1088/0004-637X/795/1/64}{Exoplanet Population Inference and the Abundance of Earth Analogs from Noisy, Incomplete Catalogs}, \apj, \textbf{795}, 64 (\arxiv{1406.3020}) [\href{http://adsabs.harvard.edu/abs/2014ApJ...795...64F}{80 citations}]

\item[{\color{numcolor}\scriptsize5}] Dawson, Rebekah I., Johnson, John Asher, Fabrycky, Daniel C., \textbf{Foreman-Mackey, Daniel}, \etal, 2014, \doi{10.1088/0004-637X/791/2/89}{Large Eccentricity, Low Mutual Inclination: The Three-dimensional Architecture of a Hierarchical System of Giant Planets}, \apj, \textbf{791}, 89 (\arxiv{1405.5229}) [\href{http://adsabs.harvard.edu/abs/2014ApJ...791...89D}{31 citations}]

\item[{\color{numcolor}\scriptsize4}] Dorman, Claire E., Widrow, Lawrence M., Guhathakurta, Puragra, Seth, Anil C., \etal\ (incl.\ \textbf{DFM}), 2013, \doi{10.1088/0004-637X/779/2/103}{A New Approach to Detailed Structural Decomposition from the SPLASH and PHAT Surveys: Kicked-up Disk Stars in the Andromeda Galaxy?}, \apj, \textbf{779}, 103 (\arxiv{1310.4179}) [\href{http://adsabs.harvard.edu/abs/2013ApJ...779..103D}{24 citations}]

\item[{\color{numcolor}\scriptsize3}] Brewer, Brendon J., \textbf{Foreman-Mackey, Daniel}, \& Hogg, David W., 2013, \doi{10.1088/0004-6256/146/1/7}{Probabilistic Catalogs for Crowded Stellar Fields}, \aj, \textbf{146}, 7 (\arxiv{1211.5805}) [\href{http://adsabs.harvard.edu/abs/2013AJ....146....7B}{16 citations}]

\item[{\color{numcolor}\scriptsize2}] \textbf{Foreman-Mackey, Daniel}, Hogg, David W., Lang, Dustin, \& Goodman, Jonathan, 2013, \doi{10.1086/670067}{emcee: The MCMC Hammer}, \pasp, \textbf{125}, 306 (\arxiv{1202.3665}) [\href{http://adsabs.harvard.edu/abs/2013PASP..125..306F}{1254 citations}]

\item[{\color{numcolor}\scriptsize1}] Weisz, Daniel R., Fouesneau, Morgan, Hogg, David W., Rix, Hans-Walter, \etal\ (incl.\ \textbf{DFM}), 2013, \doi{10.1088/0004-637X/762/2/123}{The Panchromatic Hubble Andromeda Treasury. IV. A Probabilistic Approach to Inferring the High-mass Stellar Initial Mass Function and Other Power-law Functions}, \apj, \textbf{762}, 123 (\arxiv{1211.6105}) [\href{http://adsabs.harvard.edu/abs/2013ApJ...762..123W}{23 citations}]
    \end{list}

    \cvheading{Preprints \& white papers}
    \begin{list}{}{\cvlist}
    \item[{\color{numcolor}\scriptsize10}] Hogg, David W., \& \textbf{Foreman-Mackey, Daniel}, 2017, Data analysis recipes: Using Markov Chain Monte Carlo (\arxiv{1710.06068})

\item[{\color{numcolor}\scriptsize9}] Wang, Dun, Hogg, David W., \textbf{Foreman-Mackey, Daniel}, \& Sch{\"o}lkopf, Bernhard, 2017, A pixel-level model for event discovery in time-domain imaging (\arxiv{1710.02428})

\item[{\color{numcolor}\scriptsize8}] Grunblatt, Samuel K., Huber, Daniel, Gaidos, Eric, Lopez, Eric, \etal\ (incl.\ \textbf{DFM}), 2017, Seeing double with K2: Testing re-inflation with two remarkably similar planets around red giant branch stars (\arxiv{1706.05865}) [\href{http://adsabs.harvard.edu/abs/2017arXiv170605865G}{3 citations}]

\item[{\color{numcolor}\scriptsize7}] Angus, Ruth, Morton, Timothy, Aigrain, Suzanne, \textbf{Foreman-Mackey, Daniel}, \& Rajpaul, Vinesh, 2017, Inferring probabilistic stellar rotation periods using Gaussian processes (\arxiv{1706.05459})

\item[{\color{numcolor}\scriptsize6}] Montet, Benjamin T., Tovar, Guadalupe, \& \textbf{Foreman-Mackey, Daniel}, 2017, Long Term Photometric Variability in Kepler Full Frame Images: Magnetic Cycles of Sun-Like Stars (\arxiv{1705.07928}) [\href{http://adsabs.harvard.edu/abs/2017arXiv170507928M}{2 citations}]

\item[{\color{numcolor}\scriptsize5}] \textbf{Foreman-Mackey, Daniel}, Agol, Eric, Ambikasaran, Sivaram, \& Angus, Ruth, 2017, Fast and scalable Gaussian process modeling with applications to astronomical time series (\arxiv{1703.09710}) [\href{http://adsabs.harvard.edu/abs/2017arXiv170309710F}{4 citations}]

\item[{\color{numcolor}\scriptsize4}] Luger, Rodrigo, Kruse, Ethan, \textbf{Foreman-Mackey, Daniel}, Agol, Eric, \& Saunders, Nicholas, 2017, An update to the EVEREST K2 pipeline: Short cadence, saturated stars, and Kepler-like photometry down to Kp = 15 (\arxiv{1702.05488}) [\href{http://adsabs.harvard.edu/abs/2017arXiv170205488L}{6 citations}]

\item[{\color{numcolor}\scriptsize3}] Brewer, Brendon J., \& \textbf{Foreman-Mackey, Daniel}, 2016, DNest4: Diffusive Nested Sampling in C++ and Python (\arxiv{1606.03757}) [\href{http://adsabs.harvard.edu/abs/2016arXiv160603757B}{2 citations}]

\item[{\color{numcolor}\scriptsize2}] Hogg, David W., Angus, Ruth, Barclay, Tom, Dawson, Rebekah, \etal\ (incl.\ \textbf{DFM}), 2013, Maximizing Kepler science return per telemetered pixel: Detailed models of the focal plane in the two-wheel era (\arxiv{1309.0653})

\item[{\color{numcolor}\scriptsize1}] Montet, Benjamin T., Angus, Ruth, Barclay, Tom, Dawson, Rebekah, \etal\ (incl.\ \textbf{DFM}), 2013, Maximizing Kepler science return per telemetered pixel: Searching the habitable zones of the brightest stars (\arxiv{1309.0654})
    \end{list}
\fi

\cvheading{Selected invited talks \& tutorials}
\begin{list}{}{\cvlist}

\item \emph{Long-period transiting planets \& their population},
    2016, Seminar, Canadian Institute of Theoretical Astrophysics.

\item \emph{Long-period transiting planets \& their population},
    2016, Seminar, University of Birmingham.

\item \emph{Long-period transiting planets \& their population},
    2016, Invited talk, Exoplanets I, Davos.

\item \emph{Long-period transiting planets \& their population},
    2016, Invited talk, Statistical Challenges of Modern Astrophysics,
    Carnegie Mellon.

\item \emph{Long-period transiting planets \& their population}, 2016,
    Colloquium, Villanova.

\item \emph{Long-period transiting planets \& their population},
    2016, Seminar, University of Auckland.

\item \emph{Long-period transiting planets \& their population},
    2016, Seminar, Sydney Institute for Astronomy.

\item \emph{Long-period transiting planets \& their population},
    2016, Seminar, Princeton.

\item \emph{Scalable Gaussian processes \& the search for transiting
    exoplanets}, 2015, Data Science at the LHC, CERN, Geneva.

\item \emph{Discovery \& characterization of transiting exoplanets \& their
    population}, 2015, Colloquium, University of Washington.

\item \emph{Hierarchical inference for exoplanet population inference},
    2015, IAU Symposium, Honolulu.

\item \emph{Data-driven models}, 2015, Extreme precision radial velocities,
    Yale.

\item \emph{Population inference from noisy \& incomplete catalogs}, 2015,
    Local Group Astrostatistics, University of Michigan.

\item \emph{The search for single transits},
    2015, Sagan Fellows Symposium, Caltech.

\item \emph{Inferring exoplanet populations from noisy, incomplete catalogs},
    2015, TESS group meeting, MIT.

\item \emph{Inferring exoplanet populations from noisy, incomplete catalogs},
    2015, Institute for Advanced Study, Princeton.

\item \emph{Licenses in the wild},
    2015, AAS225, Seattle.

\item \emph{Time series analysis, Gaussian Processes, and the search for
            exo-Earths},
    2014, PyData NYC conference, New York.

\item \emph{An astronomer's introduction to Gaussian processes},
    2014, Astronomy Department, University of Texas, Austin.

\item \emph{Introduction to Gaussian Processes, probabilistic graphical
            models, and deep learning},
    2014, Astro Hack Week, University of Washington.

\item \emph{Inferring exoplanet populations from noisy, incomplete catalogs},
    2014, Physics Department, University of Delaware.

\item \emph{Hierarchical inference for astronomers},
    2014, Strasbourg Observatory, France.

\item \emph{An astronomer's introduction to Gaussian processes},
    2014, Bayesian Computing for Astronomical Data Analysis (Summer school at
    Penn State University).

\item \emph{An astronomer's introduction to Gaussian processes},
    2014, Harvard--Smithsonian Center for Astrophysics.

\item \emph{Practical data analysis using MCMC},
    2014, Astronomy Department, University of Hertfordshire.

\item \emph{Practical data analysis using MCMC},
    2013, Astronomy Department, UCSC.

\item \emph{From pixels to aliens (Public Talk)},
    2013, Astronomy on Tap, NYC.

\item \emph{Data analysis using MCMC},
    2013, Astronomy Department, Columbia University.

\item \emph{Data analysis using MCMC},
    2013, Physics Department, Vanderbilt University.
\end{list}

\cvheading{Popular open-source software}
\begin{list}{}{\cvlist}

\item {\bf emcee} ---
    MCMC sampling in Python. Popular in astronomy;
    the paper has 524 citations as of 2016-01-26.
    \url{dfm.io/emcee}

\item {\bf George} ---
    Blazingly fast Gaussian processes for regression. Implemented in C++ and
    Python bindings. Joint work with applied mathematicians at NYU.
    \url{dfm.io/george}

\item {\bf corner.py} ---
    Simple corner plots (or scatterplot matrices) in Python.
    \url{github.com/dfm/corner.py}

\end{list}

\cvheading{Grants}
\begin{list}{}{\cvlist}
\item
NSF-AAG (PI: Agol),
\emph{Collaborative Research: Masses and architectures of (potentially
    habitable) exoplanet systems},
\$491,950, 2016--2018
\item
K2 Guest Observer -- Cycle 3 (PI: Penny),
\emph{Free-Floating and Bound Planet Mass Measurements with K2: Ground- and
Space-Based Photometry, Event Detection and Modeling},
\$84,000, 2016--2017
\item
K2 Guest Observer -- Cycle 3 (PI: Hogg),
\emph{Ultra-precise photometry in crowded fields: A self-calibration
approach},
\$100,000, 2016--2017
\item
XSEDE (PI: Foreman-Mackey),
\emph{A systematic search for transiting exoplanets using K2},
100,000 CPU hours, 2015--2016
\end{list}


\cvheading{Honors}
\begin{list}{}{\cvlist}

\item Kavli Fellow, 2015.
\item Sagan Postdoctoral Fellowship, 2015--present.
\item James Arthur Graduate Fellowship, 2014.
\item Horizon Fellowship in the Natural \& Physical Sciences, 2012.
\item Henry M. MacCracken Fellowship, 2010.
\item NSERC Undergraduate Summer Research Award, 2007.

\end{list}

% \ifdefined\withpubs
%     \newpage
% \fi

\cvheading{Professional service \& activities}
\begin{list}{}{\cvlist}
\item MAST Users Group --- Member
\item American Astronomical Society --- Full Member
\item Active Referee ---
    AAS Journals,
    Journal of Statistical Software,
    Journal of Open Source Software
\end{list}

\end{document}
